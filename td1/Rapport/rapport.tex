\documentclass[a4paper,11pt]{article}
\usepackage[T1]{fontenc}
\usepackage[utf8]{inputenc}
\usepackage[francais]{babel}
\usepackage[]{graphicx}
\usepackage{array}
\usepackage{amsmath}

\author{Zakaridia - Dalila}
\title{Rapport TP Coalescence}

\begin{document}
\maketitle
Ce projet a été réalisé en utilisant le logiciel octave.

\section{Prise en main des programmes}

\begin{itemize}

\item Tableau récapitulant le vecteur moyenne et matrice de variance-covariance pour les trois classes. \\\\
	\begin{tabular}{|c|c|c|}
		\hline
		classe  & matrice variance - covariance & Vecteur moyenne 
	\\
		\hline % Une ligne  
		1 &
		  $\begin{pmatrix} 4 \\ 4 \end{pmatrix}$
		& $\begin{pmatrix} 1 & 0\\ 0 & 1 \end{pmatrix}$ \\	
		
		
				\hline % Une ligne  
		2 &
		  $\begin{pmatrix} -4 \\ -4 \end{pmatrix}$
		& $\begin{pmatrix} 4 & 0\\ 0 & 4 \end{pmatrix}$ \\	
		
		
				\hline % Une ligne  
		3 &
		  $\begin{pmatrix} 0 \\ 0 \end{pmatrix}$
		& $\begin{pmatrix} 1 & 0\\ 0 & 1 \end{pmatrix}$ \\	
		\hline
		
		
	\end{tabular} \\

\item
	La fonction $[x$ $m$ $clas] = initialiser()$ permet de générer aléatoirement les données de test et d'initialiser la partition optimale. Elle renvoie également la métrique associée aux données de test. En utilisant la fonction $affiche\_classe(x,clas)$ , nous obtenons la représentation graphique suivante représentant celle de la partition optimale. 

\item
L'algorithme de coalescence a été codé dans le fichier $coalescence.m$.

\item
Pour tester l'algorithme, nous procédons comme suite : \\
	\begin{itemize}
		\item Initialisation des données : $[x$ $m$ $clas] = initialiser()$
		\item Exécution de la fonction $[clas2$ $g]$ = $lancerAleatoire(x,k,m)$ : Dans cette fonction, nous choisissons aléatoirement les centres de gravité initiaux puis on exécute l'algorithme de coalescence. $clas2$ contient le résultat de la classification et $g2$ le centre de gravité par classe. 
	\end{itemize}
	
\item
Après exécution de l'algorithme de coalescence et affichage du résultat, nous observons des erreurs de classifications dues à la dispersion des individus au sein des classes ( notamment la dispersion des individus dans la partition optimale ayant la classe numéro 2 ) . 

\end{itemize}

\section{Influence du point de l’initialisation}

	\begin{itemize}
		\item Chargement des données : $[x$ $m]$ = $charger("td2\_d1.txt")$ \\
		\item Visualisation planaire des données : $visualiser(x)$ \\
		\item La visualisation planaire des individus nous permet de constater que ceux-ci peuvent être regrouper en deux classes. \\
		En regroupant donc ces individus par application de l'algorithme de coalescence, nous observons que la répartition obtenu ne nous convient pas a chaque fois.\\
		Le problème se situe lors de l'initialisation des centres de gravité. En effet choisir les deux centres gravités parmi les individus qui se ressemblent le plus, conduit à une partition non satisfaisante.\\
		\item Pour choisir des centres de gravité permettant d'obtenir une partition satisfaisante, nous choisissons deux individus en fonction de la moyenne (m) et de l'ecart-type (e) de la deuxième mesure. Les deux  valeurs devront respecter les contraintes suivantes :
		\begin{itemize}
			\item  La première valeur $\ge e + m$ 
			\item La seconde valeur $\le m - e$ 
		\end{itemize}
	\end{itemize} 

\section{Influence de la métrique}


\begin{itemize}
		\item Chargement des données : $[x$ $m]$ = $charger("td2\_d.txt")$ \\
		\item Visualisation planaire des données : $visualiser(x)$ \\
		\item La visualisation planaire des individus nous permet de constater que ceux-ci peuvent être regrouper en deux classes. \\
             En regroupant donc ces individus par application de l'algorithme de coalescence, nous observons que la répartition obtenu ne nous convient pas a chaque fois.\\
		Le problème se situe lors de l'initialisation des centres de gravité.Et pour palier a ce probleme nous proposons une solution qui consiste a utilier la meyenne de la premiere et la deuxieme          mesure 
             pour calculer les deux mesures associers aux deux centre  de gravite G1 et G2  .
\begin{itemize}
             \item Calculer la moyene de la premiere mesure Moym1 .
              \item Calculer la moyene de la deuxieme mesure Moym2.
          
\begin{itemize}
              \item G1 (Moym1 ,  (min(x(2,:)) + moyM2)/2)).
               \item G2 (Moym2 ,  (max(x(2,:)) + moyM2)/2).
  \end{itemize}

 \end{itemize}
         le resultat obtenu est très satisfaisant.
         \item  avec l'ecart type de la mesure 1 est de 0.1 et que l'ecart type de la mesure 2 et de 2 on appliquant la solution precedente nous obtenant un resultat insatisfaisant ,donc la metrique a bien influencer sur le resulat.
         La solution proposé est consiste à calculer la metriqe de la façon suivante: 
         \begin{itemize}
              \item M(1,1)=1/VAR(x(1,:)).
               \item M(2,2)=1/VAR(x(2,:)).
       Le resultat obtenu est tres satisfaisant .
  \end{itemize}
		\end{itemize}
		
\section{Choix du nombre de classes}
	L'algorithme de coalescence vise à minimiser la valeur de l'inertie intra-classe. 
D'après la courbe de représentation de cette valeur en fonction du nombre de classe, nous constatons que celle-ci décroît très peu à partir du nombre de classe K = 4. Nous pouvons donc conclure que le nombre de classe de l'ensemble d’apprentissage est 4. 
	\includegraphics[width=200px]{../images/courbe.png}
\end{document}